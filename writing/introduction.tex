%!TEX root = splineBART_main.tex
For each of $i = 1, \ldots, n$ surfaces, we observe (i) a vector of surface parameters $\bz_{i}$ and (ii) $n_{i}$ pairs of wall superheat -- flux  $(x_{i1}, y_{i1}), \ldots (x_{in_{i}}, y_{in_{i}}).$
At a high level, we model
$$
y_{it} = F(x_{it}; \bz_{i}) + \sigma \varepsilon_{it}; \quad \varepsilon_{it} \sim \N(0,1).
$$
Our main interest is predicting the \textit{boiling curve} as a function of the surface parameters $\bz_{i},$ which include both the sandblasting parameters and local surface roughness measurements.
That is, we would like to estimate the entire curve $F(\cdot ; \bz)$ for each $\bz.$

To this end, we approximate $F(\cdot, \bz)$ using a linear combination of $D$ pre-specified basis functions $\varphi_{1}(\cdot), \ldots, \varphi_{D}(\cdot).$
For each $x$ in the range of observable wall superheats, let $\bphi(x)^{\top} = (\varphi_{1}(x), \ldots, \varphi_{D}(x))$ be the vector of all basis function evaluations at $x.$
We adopt a \textit{varying coefficient} model for the boiling curve $F(\cdot, \bz)$:
\begin{equation}
\label{eq:varying_coef_spline}
F(x; \bz) = \bphi(x)^{\top}\bbeta(\bz)
\end{equation}

In other words, if we let $\Phi_{i}$ be the $n_{i} \times D$ whose $t^{\text{th}}$ row is $\bphi(x_{it})^{\top},$ then we are modeling
$$
\by_{i} \sim \N_{n_{i}}\left(\Phi_{i}\bbeta(\bz_{i}), \sigma^{2}I_{n_{i}}\right),
$$
independently for all $i = 1, \ldots, n.$

In light of our working model in Equation~\eqref{eq:varying_coef_spline}, our main goal is to estimate the vector-valued function $\bbeta(\bz).$
In this note, we develop a new method based that extends \citet{Chipman2010}'s Bayesian Additive Regression Trees (BART) framework to this setting.
The proposed method can also be seen as a natural extension of \citet{Low-Kam2015}'s single tree model.
